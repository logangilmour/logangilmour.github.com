\documentclass[10pt]{article}
\usepackage[margin=2cm]{geometry}
\usepackage{array, xcolor}
\definecolor{lightgray}{gray}{0.8}
\newcolumntype{L}{>{\raggedleft}p{0.2\textwidth}}
\newcolumntype{R}{p{0.74\textwidth}}
\newcommand\VRule{\color{lightgray}\vrule width 0.5pt}
\setlength{\parindent}{0pt}
\title{\bfseries\Large Logan Gilmour}
\author{}
\date{}
\begin{document}
\maketitle
\noindent 9506 72 Ave NW \hfill 1 (780) 999-1998\\
Edmonton, AB  T6E 0Y4 \hfill legilmou@ualberta.ca
\section*{Education}

B.Sc. Honours Computing Science\\
June 12, 2013, with First Class Honours\\
University of Alberta, AB, CA\\
GPA: 3.48, Major: 3.60\\
\\
M.Sc. Computing Science\\
January 2019 - Present\\
University of Alberta, AB, CA\\
Current GPA: 3.93\\

\section*{Experience}

\subsection*{Self Employed}
September 2020 - Present

Refining photogrammetry software for a pilot project funded by SSRIA in collaboration with Butterwick Projects Ltd. Providing geometric reconstructions of three single-family homes with photographs shot via drone using the developed photogrammetry software. Accepted into imYEG commercialization accelerator.

\subsection*{Graduate TA/RA}
January 2019 - Present

\emph{University of Alberta}
\vspace{\baselineskip}

Responsible for grading assigments and exams, and assisting students during office hours. Conducting research on methods for reconstructing geometry from images, with a focus on building facades.

\subsection*{Mitacs Accelerate Intern}
January 2020 - May 2020

\emph{Butterwick Projects Ltd. via University of Alberta}
\vspace{\baselineskip}

Worked to ascertain feasibility of a photogrammetry-based approach for developing as-built 3D models of buildings to support prefabricated exterior energy retrofits. Conducted a literature review of building reconstruction techniques, operated cameras on site, worked with existing photogrammetry software, and built a proof-of-concept software parametrizing the reconstruction optimization in terms of planar rectified geometry.

\subsection*{Cofounder, Software Developer, Game Designer}
January 2015 - Present

\emph{ThirtyThree}
\vspace{\baselineskip}

Built the game RunGunJumpGun for PC/Mac/iOS/Android/Nintendo Switch using Unity3D. Designed a cellular automata system for procedurally aided level design. Developed a palette-based sprite rendering system. Built a custom collision detection system. Designed many of the game's levels. Wrote code to procedurally generate art assets. Implemented a promotional website. Integrated with Steam, iOS, Android, and Mac App store for achievements and leaderboards. Added analytics and wrote R scripts for statistical analysis of user playthroughs to improve difficulty progression. Built a prototype Virtual Reality Game with funding from the Canadian Media Fund and Alberta Interactive Digital Media Grant. Designed levels, built volumetric light system optimized for VR with single-buffer temporal stereo reprojection, designed gameplay systems.


\subsection*{Graphics Programmer}
December 2014 - May 2016, January 2017 - September 2017

\emph{ScopeAR}
\vspace{\baselineskip}

Built several Augmented Reality step-by-step instructional applications using an in-house tool built on top of Unity3D. Rebuilt the animation system at the core of that tool to create a robust and intuitive authoring tool for step-by-step instructional content that is targeted toward users of limited technical ability. Designed a variety of shaders and rendering techniques that can highlight or deemphasize content in an aesthetically pleasing way, and a system for applying these visibility modifiers in a hierarchical way. Developed a content management system that accesses and downloads that content via the internet. Developed a technique for drawing on a mesh generated from a depth-camera point-cloud in real-time Augmented Reality. Built a pipeline for encoding and transmitting video mixed with augmented reality data in real-time.

\subsection*{Software Developer}
Jan 2014 - December 2014

\emph{Sticks \& Stones}
\vspace{\baselineskip}

Developed a variety of interactive websites using Javascript, CSS, and HTML. Built several custom Wordpress deployments. Designed and developed a prototype networked collaborative drawing tool.

\subsection*{Undergraduate Research Assistant}
May 2012 - September 2013

\emph{Service Systems Research Group, University of Alberta, Dr. Eleni Stroulia, Dr. Sarah Forgie}
\vspace{\baselineskip}

Designed and implemented a prototype web-based e-learning tool that logs all user interactions using Clojure, ClojureScript, CouchDB, and Tomcat. Developed a prototype tool that generates interactive HTML from RDF. Full-time during the summers of 2012 and 2013, and part-time during the school year.

\subsection*{Teaching Assistant}
Jan 2012 - May 2012

\emph{CMPUT 297: Introduction to Tangible Computing II, University of Alberta, Dr. James Hoover, Dr. Michael Bowling}
\vspace{\baselineskip}

Aided students in learning course content during office hours, and graded programming quizzes and assignments.

\subsection*{Software Development Intern}
May 2011 - Dec 2011

\emph{Health Care Aids and Technology project funded by Alberta Health and Wellness, Dr. Lili Liu}
\vspace{\baselineskip}

Designed and implemented a prototype application for scheduling Health Care Aids, using Google Web Toolkit, Tomcat and PostgreSQL for scheduling application and PhoneGap for a mobile app for schedule viewing and task completion. Also adapted this software to schedule medication reminders for a smart-pillbox based on an Arduino microcontroller.

\subsection*{Undergraduate Research Assistant}
May 2010 - Dec 2010

\emph{The Service Systems Research Group, University of Alberta, Dr. Margaret Mackey, Dr. Eleni Stroulia}
\vspace{\baselineskip}

Designed and implement an XML-based language for developing iPhone-based interactive spatial ebooks in Objective-C, and developed a demo e-book using the language I developed. Designed and implemented a declarative language for web-scraping in Java. Developed a small tool to aid in digitizing health-care forms.



\subsection*{Self Employed}
May 2010 - December 2013
\vspace{\baselineskip}

Built several web-pages using PHP on shared Apache hosting. Built two web-applications with GWT running on Tomcat hosted on Amazon EC2.

\section*{Selected Projects}

RunGunJumpGun for PC/Mac/iOS/Android (Video Game). http://rungunjumpgun.com \hfill 2016 \\

``The Box'' (VR Experience). https://vimeo.com/201581136 \hfill 2016 \\

Moustache of Ceremonies (Digital Puppet). https://vimeo.com/202081740 \hfill 2015 \\

Kasketball for PC/Mac (Video Game). https://thirtythreegames.itch.io/kasketball \hfill 2015 \\

\section*{Exhibitions}
``The Box'', \emph{Game Start}, Latitude Art Gallery, Edmonton, AB \hfill 2016 \\
\section*{Publications}
Gilmour, L.; Ray N.;, ``Locating Cephalometric X-Ray Landmarks with Foveated Pyramid Attention,'' \emph{Medical Imaging with Deep Learning} (pp. 262-276) PMLR 2020.\\

Abbey, B.; Alipour, A.; Gilmour, L.; Camp, C.; Hofer, C.; Lederer, R.; Rasmussen, G.; Lili Liu; Nikolaidis, I.; Stroulia, E.; Sadowski, C.; , ``A remotely programmable smart pillbox for enhancing medication adherence,'' \emph{Computer-Based Medical Systems (CBMS), 2012 25th International Symposium on}, pp.1-4, 20-22 June 2012



Gilmour,

\section*{Awards and Recognition}

\noindent iOS App Store Editor's Choice - \emph{RunGunJumpGun} \hfill 2016 \\

\noindent Bit Bash 2016 Official Selection - \emph{Kasketball}, Chicago, IL \hfill 2016 \\

\noindent ACE Award ‘Innovative Use of Technology’ - \emph{Moustache of Ceremonies}, Edmonton, AB \hfill 2015 \\

\noindent NSERC Undergraduate Summer Research Award \hfill 2012 \\

\noindent Dean's Honor Roll \hfill 2010-2011, 2011-2012, 2012-2013 \\

\noindent Jason Lang Scholarship \hfill 2010, 2011 \\

\vspace{\baselineskip}

\noindent References available upon request

\end{document}
